\documentclass{article}

\usepackage{fullpage}
\usepackage{graphicx}
\usepackage[backend=biber]{biblatex}

\title{Programmeerproject 2: Voorstudie}
\author{Jannick Hemelhof}

\begin{document}

\maketitle
\newpage
  \tableofcontents
\newpage
\pagenumbering{arabic}

\section{Overzichtsdiagram}

\begin{figure}[hb]%                 use [hb] only if necceccary!
  \centering
  \includegraphics[scale=0.7]{overzichtsdiagram}
  \caption{Overzichtsdiagram}
  \label{fig:test}
\end{figure}
De pijlen hebben volgende betekenis:
\begin{itemize}
  \item zwarte pijl met gevulde punt "x $\rightarrow$ y" : Dit wil zeggen dat x gebruikt maakt van y, x doet dus beroep op de 'make'-functie van y en vervolgens met dit object zal werken (er mee communiceren).
  \item groene pijl met gevulde punt "x $\rightarrow$ y" : Dit wil zeggen dat x enkel en alleen communiceert met y, het geeft dus aan hoe de belangrijkste lijnen van de communicatie lopen.
  \item zwarte pijl met lege punt "x $\rightarrow$ y" : Deze pijl heeft als betekenis dat x het object y alleen maar zal aanmaken. Het wordt eventueel meegegeven aan andere objecten die x zal aanmaken maar er wordt niet (rechtstreeks) gecommuniceerd met y.
\end{itemize}

\section{Bespreking ADTs}

\subsection{route} % OK
Een ADT dat instaat voor het beheer van een route over meerdere segmenten en/of wissels heen. Houdt bij welke segmenten we moeten bereiken om vanuit onze origin naar de destination te geraken. Bij aanmaak worden twee argumenten meegegeven met een optioneel derde argument:
\begin{itemize}
  \item origin: Van waar vertrekt deze route?
  \item destination: Naar waar gaat deze route?
  \item path: Optioneel derde argument in de vorm van een lijst met te volgen segmenten en/of wissels. Geeft aan welk pad de route exact moet volgen.
\end{itemize}
Indien we geen derde argument meegeven \textit{kan} er gebruik worden gemaakt van een optimaal pad dat wordt gezocht tussen origin en destination.
Volgende boodschappen worden door het ADT ondersteund:
\begin{center}
    \begin{tabular}{ | l | l | l | p{8cm} |}
    \hline
    Commando & Argument(en) & Return value & Omschrijving \\ \hline
    getOrigin & / & origin (string) & Geeft de oorsprong van de route terug in de vorm van een string. \\ \hline
    getDestination & / & destination (string) & Geeft de bestemming van de route terug in de vorm van een string. \\ \hline
    getLength & / & length (number) & Geeft de lengte van de route terug in de vorm van een number. \\ \hline
    detected! & / & / & Geeft aan dat we het volgende detectieblok voorbij zijn en dat we dit hier ook moeten aanpassen. Op deze manier weten we hoe ver we gevorderd zijn op de route en kunnen we indien nodig een trein die het kortst bij zijn bestemming is een hogere prioriteit geven bij een obstakel/ongeval. \\ \hline
    getTodo & / & todo (list) & Zorgt ervoor dat we een lijst terugkrijgen met daarin de route tot aan het volgende detectieblok. \\ \hline
    \end{tabular}
\end{center}

\noindent Met het \textit{getTodo} commando kan het NMBS-gedeelte eenvoudig bepalen welke commando's naar het Infrabel-gedeelte moeten gestuurd worden om de route voor een zekere trein mogelijk te maken.

\subsection{timer} % OK
Dankzij dit ADT kunnen we de notie van tijdsevolutie gebruiken in het project. Bij aanmaak worden drie argumenten meegegeven:
\begin{itemize}
  \item timerLength: Hoelang later we onze timer steeds lopen?
  \item object: Naar welk object zullen we de boodschap sturen telkens de timer afloopt?
  \item message: Welke boodschap sturen we naar het object als de timer telkens de timer afloopt?
\end{itemize}
Een timer volgt dus volgend principe:
\begin{enumerate}
  \item Timer wordt aangemaakt. Is nog niet actief.
  \item Timer ontvangt boodschap 'start'
  \item Timer begint te lopen vanaf timerLength tot 0
  \item 0 wordt bereikt, stuur message naar het object. Ga opnieuw naar 3.
\end{enumerate}
Dit wordt steeds herhaald tot de timer de boodschap 'stop' ontvangt. Volgende boodschappen worden door het ADT ondersteund:
\begin{center}
    \begin{tabular}{ | l | l | l | p{8cm} |}
    \hline
    Commando & Argument(en) & Return value & Omschrijving \\ \hline
    start & / & / & Laat de timer starten \\ \hline
    stop & / & / & Laat de timer stoppen \\ \hline
    setTimerLength! & timerLength (number) & / & Past de lengte van de timer aan naar de meegegeven lengte. \\ \hline
    setObject! & object (object) & / & Stuur de boodschap naar een ander object, zijnde het meegegeven object. \\ \hline
    setMessage! & message (string) & / & Stuur een andere boodschap naar het object, zijnde de meegegeven boodschap. \\ \hline
    \end{tabular}
\end{center}

\subsection{signal} % OK
Dit ADT stelt een licht voor dat zich kan bevinden op een segment. Het houdt een status bij en stelt andere objecten in staat om deze status op te vragen en aan te passen. Bij aanmaak wordt \'{e}\'{e}n argument meegegeven:
\begin{itemize}
  \item id: Uniek ID dat het licht identificeert
\end{itemize}
Volgende boodschappen worden door het ADT ondersteund:
\begin{center}
    \begin{tabular}{ | l | l | l | p{8cm} |}
    \hline
    Commando & Argument(en) & Return value & Omschrijving \\ \hline
    getID & / & id (string) & Geeft het ID van het licht terug in de vorm van een string. \\ \hline
    getStatus & / & status (string) & Geeft de status van het licht terug in de vorm van een string. \\ \hline
    setStatus! & status (string) & / & Past de status van het licht aan naar de meegegeven status. \\ \hline
    \end{tabular}
\end{center}

\subsection{logger} % OK
Een ADT dat zal instaan voor het loggen - in dit geval wegschrijven naar een bestand - van de belangrijke gebeurtenissen die plaatsvinden tijdens het uitvoeren van de simulatie. Bij aanmaak worden geen argumenten meegegeven.
Volgende boodschappen worden door het ADT ondersteund:
\begin{center}
    \begin{tabular}{ | l | p{3.5cm} | l | p{8cm} |}
    \hline
    Commando & Argument(en) & Return value & Omschrijving \\ \hline
    addTrain & id (string) & / & Logt dat er een nieuwe trein 'id' is toegevoegd. \\ \hline
    segmentPassed & trainID (string) + segmentID (string) & / & Logt dat een trein 'trainID' een segment 'segmentID' is gepasseerd. \\ \hline
    switchPassed & trainName (string) + switchName (string) & / & Logt dat een trein 'trainID' een wissel 'switchID' is gepasseerd. \\ \hline
    speedChanged & trainID (string) + newSpeed (number) & / & Logt dat een trein 'trainID' vertraagt/versnelt tot snelheid 'newSpeed'. \\ \hline
    directionChanged & trainID (string) + newDirection (number) & / & Logt dat een trein 'trainID' naar richting 'newDirection' is veranderd. \\ \hline
    trainStopped & trainID (string) + reason (string) & / & Logt een een trein 'trainID' gestopt is vanwege 'reason'. \\ \hline
    switchChanged & switchID (string) + newPosition (number) & / & Logt dat een switch 'switchID' schakelt naar positie 'newPosition'. \\ \hline
    signalChanged & signalID (string) + newStatus (string) & / & Logt dat een signaal 'signalID' een nieuwe status 'newStatus' heeft. \\ \hline
    messageSend & message (string) & / & Logt dat er een commando werd verstuurd naar Infrabel \\ \hline
    getLogged & / & logged (list) & Geeft een lijst terug met daarin alle acties die werden gelogd sinds de laatste keer dat dit commando werd opgeroepen. \\ \hline
    \end{tabular}
\end{center}

\noindent Een apart commando is beschikbaar in de logger om aan te geven dat er een bericht werd verstuurd naar Infrabel. Aangezien in deel 2 deze via het netwerk moet bereikt worden is het handig om te weten of er wel degelijk een commando over het netwerk werd gestuurd.

\iffalse
\subsection{position}
Een klein ADT dat een positie zal voorstellen. In dit geval op een x- en y-as maar dit kan makkelijk worden aangepast. Bij aanmaak worden twee argumenten meegegeven:
\begin{itemize}
  \item x: Positie op de x-as
  \item y: Positie op de y-as
\end{itemize}
Volgende boodschappen worden door het ADT ondersteund:
\begin{center}
    \begin{tabular}{ | l | l | l | p{8cm} |}
    \hline
    Commando & Argument(en) & Return value & Omschrijving \\ \hline
    getX & / & x (number) & Geeft de positie op de x-as terug in de vorm van een number. \\ \hline
    getY & / & y (number) & Geeft de positie op de y-as terug in de vorm van een number. \\ \hline
    setX! & newX (number) & / & Past de positie op de x-as aan naar de gegeven positie. \\ \hline
    setY! & newY (number) & / & Past de positie op de y-as aan naar de gegeven positie. \\ \hline
    \end{tabular}
\end{center}
\fi

\newpage
\subsection{locomotive} % OK
Het ADT dat een locomotief zal voorstellen en ook zal bijhouden welke wagons er eventueel aan vasthangen. Bij aanmaak worden twee argumenten meegegeven met een optioneel derde:
\begin{itemize}
  \item id: Een uniek ID voor de trein.
  \item position: Beginpositie van de trein.
  \item wagons: Optioneel derde argument dat de wagons vastgemaakt aan de locomotief bevat.
\end{itemize}
Volgende boodschappen worden door het ADT ondersteund:
\begin{center}
    \begin{tabular}{ | l | l | l | p{6.5cm} |}
    \hline
    Commando & Argument(en) & Return value & Omschrijving \\ \hline
    getID & / & id (string) & Geeft het unieke ID van de locomotief terug in de vorm van een string. \\ \hline
    getSpeed & / & speed (number) & Geeft de snelheid van de locomotief terug in de vorm van een number. \\ \hline
    setSpeed! & newSpeed (number) & / & Past de snelheid van de locomotief aan naar de meegegeven snelheid. \\ \hline
    getLastSeen & / & lastSeen (string) & Geeft het segment (detectieblok) terug waar de locomotief voor het laatst werd gezien. \\ \hline
    updateLastSeen! & lastSeen (string) & / & Past aan waar de locomotief voor het laatst werd gezien. \\ \hline
    \end{tabular}
\end{center}

\subsection{wagon} % OK
Een ADT dat een wagon voorsteld. Een wagon heeft enkel een uniek ID alsook een type dat aangeeft wat de wagon precies vervoert. Bij aanmaak worden twee argumenten meegegeven:
\begin{itemize}
  \item id: Een uniek ID voor de wagon.
  \item type: Wat de wagon precies vervoert.
\end{itemize}
Volgende boodschappen worden door het ADT ondersteund:
\begin{center}
    \begin{tabular}{ | l | l | l | p{8cm} |}
    \hline
    Commando & Argument(en) & Return value & Omschrijving \\ \hline
    getID & / & id (string) & Geeft het ID van de wagon terug in de vorm van een string. \\ \hline
    getType & / & type (string) & Geeft het type van de wagon terug in de vorm van een string. \\ \hline
    setType! & newType (string) & / & Past het type van de wagon aan naar het meegegeven type. \\ \hline
    getLoad & / & amount (int) & Geeft aan hoezeer de wagon gevult is \\ \hline
    load! & amount (int) & / & Vult de wagon met een bepaalde hoeveelheid ('amount') van zijn type \\ \hline
    unload! & amount (int) & / & Haalt een bepaalde hoeveelheid ('amount') uit de wagon \\ \hline
    \end{tabular}
\end{center}

\newpage

\subsection{switch} % OK
Dit ADT stelt een wissel voor. Er wordt bijgehouden welke segmenten met de wissel bereikbaar zijn, wat de huidig status is maar ook eventueel in welke volgorde een schakeling moet gebeuren. Bij aanmaak worden drie argumenten meegegeven met een optioneel vierde:
\begin{itemize}
  \item id: Een uniek ID voor de wissel.
  \item segments: Een lijst (minimale lengte 3) met de mogelijke segmenten die bereikbaar zijn met deze wissel.
  \item default: Welke status wordt gezien als de "default" (d.w.z. conceptueel gezien rechtdoor)?
  \item switchOrder: Optioneel vierde argument dat aangeeft hoe er tussen de verschillende mogelijke statussen geschakeld moet worden.
\end{itemize}
Volgende boodschappen worden door het ADT ondersteund:
\begin{center}
    \begin{tabular}{ | l | l | l | p{8cm} |}
    \hline
    Commando & Argument(en) & Return value & Omschrijving \\ \hline
    getID & / & id (string) & Geeft het ID van de wissel terug in de vorm van een string. \\ \hline
    getSegments & / & segments (list) & Geeft alle bereikbare segmenten hun ID terug in de vorm van een lijst. \\ \hline
    getActive & / & active (list) & Geeft de huidig actieve status terug in de vorm van een lijst met daarin de IDs van bereikbare segmenten. \\ \hline
    setActive! & mode (list) & / & Laat de wissel schakelen naar de gegeven mode. \\ \hline
    \end{tabular}
\end{center}

\subsection{segment} % OK
Een ADT dat een normaal treinsegment zonder wissels voorstelt. Er wordt bijgehouden welke segmenten bereikbaar zijn en of er eventueel lichten op het segment aanwezig zijn. Bij aanmaak worden twee argumenten meegegeven met een optioneel derde en vierde:
\begin{itemize}
  \item id: Een uniek ID voor het segment.
  \item segments: Een lijst met de bereikbare segmenten.
  \item signal: Optioneel derde argument in de vorm van een signal-object dat aangeeft of er een licht op het segment aanwezig is.
  \item detection: Optioneel vierde argument dat aangeeft of er een detectieblok op dit segment aanwezig is.
\end{itemize}
Volgende boodschappen worden door het ADT ondersteund:
\begin{center}
    \begin{tabular}{ | l | l | l | p{6cm} |}
    \hline
    Commando & Argument(en) & Return value & Omschrijving \\ \hline
    getID & / & id (string) & Geeft het ID van het segment terug in de vorm van een string. \\ \hline
    getSegments & / & segments (list) & Geeft de bereikbare segmenten terug in de vorm van een lijst. \\ \hline
    signalCommand & command (string) + extra (list) & result (string) & Zal het licht een bepaalt commando 'command' laten uitvoeren, eventueel met behulp van optionele paramaters 'extra' \\ \hline
    \end{tabular}
\end{center}

\subsection{infrabel} % OK
ADT dat specifiek instaat voor de infrastructuur, d.w.z. het aansturen van wissels en signalisatie maar ook zorgt voor het respecteren van de veiligheid (afstand tussen treinen bijvoorbeeld). Bij aanmaak worden geen argumenten meegegeven.
Volgende boodschappen worden door het ADT ondersteund:
\begin{center}
    \begin{tabular}{ | l | p{4cm} | l | p{6cm} |}
    \hline
    Commando & Argument(en) & Return value & Omschrijving \\ \hline
    addTrain & id (string) + start (string) (+ wagons (list)) & / & Voegt een trein toe. \\ \hline
    manipulateTrain & id (string) + command (string) + extra (list) & / & Zal een trein 'id' een bepaalt commando 'command' laten uitvoeren, eventueel aan de hand van optionele paramaters 'extra'. \\ \hline
    manipulateSignal & id (string) + newStatus (string) & / & Zal de status van een signaal 'id' veranderen naar de meegegeven status 'newStatus'. \\ \hline
    manipulateSwitch & id (string) + newMode (int) & / & Zal een wissel laten schakelen naar de nieuwe modus 'newMode'. \\ \hline
    getActions & / & actions (list) & Geeft een lijst terug met daarin alle uitgevoerde acties sinds de laatste keer dat dit commando werd opgeroepen. \\ \hline
    start & / & / & Zal het monitoren inschakelen. \\ \hline
    stop & / & / & Zal het monitoren uitschakelen. \\ \hline
    \end{tabular}
\end{center}

\subsection{infrabelNode} % OK
Dit ADT zal instaan voor de communicatie tussen het NMBS-gedeelte en het Infrabel-gedeelte van het programma. Hoewel in Deel 1 deze nog op hetzelfde systeem draaien is dit niet meer het geval in Deel 2 en daar willen we op voorbereid zijn. Conceptueel is het niet logisch dat het NMBS-gedeelte ook zou moeten zorgen voor netwerkcommunicatie, dus vandaar dat deze logica in een apart ADT wordt gestoken. Het ADT accepteert dezelfde commando's als het Infrabel-gedeelte. Bij aanmaak wordt \'{e}\'{e}n argument meegegeven:
\begin{itemize}
  \item logger: Een logger die het ADT zal gebruiken.
\end{itemize}
Volgende boodschappen worden door het ADT ondersteund:
\begin{center}
    \begin{tabular}{ | l | p{4cm} | l | p{6cm} |}
    \hline
    Commando & Argument(en) & Return value & Omschrijving \\ \hline
    addTrain & id (string) + start (string) (+ wagons (list)) & / & Voegt een trein toe. \\ \hline
    manipulateTrain & id (string) + command (string) + extra (list) & / & Zal een trein 'id' een bepaalt commando 'command' laten uitvoeren, eventueel aan de hand van optionele paramaters 'extra'. \\ \hline
    manipulateSignal & id (string) + newStatus (string) & / & Zal de status van een signaal 'id' veranderen naar de meegegeven status 'newStatus'. \\ \hline
    manipulateSwitch & id (string) + newMode (int) & / & Zal een wissel laten schakelen naar de nieuwe modus 'newMode'. \\ \hline
    getActions & / & actions (list) & Geeft een lijst terug met daarin alle uitgevoerde acties sinds de laatste keer dat dit commando werd opgeroepen. \\ \hline
    start & / & / & Zal het monitoren inschakelen. \\ \hline
    stop & / & / & Zal het monitoren uitschakelen. \\ \hline
    \end{tabular}
\end{center}

\newpage

\subsection{NMBS}
ADT dat instaat voor beheer van routes. Zal voor elk locomotief een route bijhouden en op deze manier bepalen welke wissels moeten schakelen. Bij aanmaak wordt \'{e}\'{e}n argument meegegeven:
\begin{itemize}
  \item logger: Een logger die het ADT zal gebruiken.
\end{itemize}
Volgende boodschappen worden door het ADT ondersteund:
\begin{center}
    \begin{tabular}{ | l | p{4cm} | l | p{6cm} |}
    \hline
    Commando & Argument(en) & Return value & Omschrijving \\ \hline
    createRoute & id (string) + start (string) + destination (string) + segments (list) & / & Wijst aan een trein 'id' een route toe tussen 'start' en 'destination' die eventueel gebruik maakt van bepaalde segmenten/wissels. \\ \hline
    getDrawable & / & drawable (list) & Geeft een weergave van de opstelling terug die de GUI kan tekenen. \\ \hline
    \end{tabular}
\end{center}

\subsection{layout}
Een ADT dat onze virtuele opstelling zal bijhouden. Encapsuleert dus een weergave van het veld, een graf van segmenten/wissels en de treinen die er op rijden. Bij aanmaak worden geen argumenten meegegeven.
Volgende boodschappen worden door het ADT ondersteund:
\begin{center}
    \begin{tabular}{ | l | p{4cm} | l | p{6cm} |}
    \hline
    Commando & Argument(en) & Return value & Omschrijving \\ \hline
    addTrain & id (string) + start (string) (+ wagons (list)) & / & Voegt een trein toe. \\ \hline
    manipulateTrain & id (string) + command (string) + extra (list) & / & Zal een trein 'id' een bepaalt commando 'command' laten uitvoeren, eventueel aan de hand van optionele paramaters 'extra'. \\ \hline
    manipulateSignal & id (string) + newStatus (string) & / & Zal de status van een signaal 'id' veranderen naar de meegegeven status 'newStatus'. \\ \hline
    manipulateSwitch & id (string) + newMode (list) & / & Zal een wissel laten schakelen naar de nieuwe modus 'newMode'. \\ \hline
    getDrawable & / & drawable (list) & Zal een lijst teruggegeven met de opbouw van het veld zodat de GUI het kan tekenen. \\ \hline
    getActions & / & actions (list) & Geeft een lijst terug met daarin alle uitgevoerde acties sinds de laatste keer dat dit commando werd opgeroepen. \\ \hline
    processActions & actions (list) & / & Past een lijst met acties toe op de opstelling. \\ \hline
    loadSchema & filename (string) & / & Laadt een opstelling vanuit het bestand 'filename'. \\ \hline
    saveSchema & / & / & Schrijft de opstelling weg naar een bestand. \\ \hline
    getIDs & / & IDs (list) & Geeft de IDs terug van alle onderdelen (segmenten, wissels en locomotieven) terug. \\ \hline
    \end{tabular}
\end{center}

\subsection{GUI}
ADT dat de interface en eigelijk het hele programma zal beheren. Bij aanmaak worden twee argumenten meegegeven:
\begin{itemize}
  \item height: Hoe hoog moet het venster zijn?
  \item width: Hoe breed moet het venster zijn?
\end{itemize}
Volgende booschappen worden door het ADT ondersteund:
\begin{center}
    \begin{tabular}{ | l | l | l | p{8cm} |}
    \hline
    Commando & Argument(en) & Return value & Omschrijving \\ \hline
    draw & rate (int) & / & Zorgt dat het venster elke 'rate' seconden opnieuw zal worden getekend. \\ \hline
    \end{tabular}
\end{center}
\noindent De GUI zal dus met behulp van een timer elke rate seconden het beeld refreshen (opnieuw tekenen). Daarvoor moet er natuurlijk eerst bepaalt worden wat er moet worden getekend. Dat zal de GUI aan het NMBS-gedeelte (verantwoordelijk voor de lay-out van de opstelling) vragen met behulp van het \textit{getDrawable} commando. Met de lijst die als return value terugkomt kan de GUI dan verder. \\
\\
Verder wordt ook user input gedetecteerd en vervolgens verwerkt. Commando's worden verstuurd naar hun verantwoordelijke ADTs (NMBS of infrabelNode). Resultaten van deze commando's worden getoond (= getekend) met de volgende refresh.

\section{Planning}
\begin{center}
    \begin{tabular}{ | l | l |}
    \hline
    Week & Omschrijving \\ \hline
    4 & Design ADTs \\ \hline
    5 & Design ADTS + schrijven voorstudie \\ \hline
    6 & Design ADTs, werken aan ADT wagon + finaliseren voorstudie \\ \hline
    7 & Afwerken ADTs timer, locomotive, wagon, signal \\ \hline
    8 & Afwerken ADTs segment, switch, route \\ \hline
    9 & Afwerken ADTs logger, layout \\ \hline
    10 & Werken aan ADTs NMBS, infrabel, infrabelNode \\ \hline
    11 & Afwerken ADTs NMBS, infrabel, infrabelNode \\ \hline
    12 & Basis van GUI en communicatie tussen GUI en NMBS/infrabelNode \\ \hline
    13 & GUI \\ \hline
    14 & GUI afwerken \\ \hline
    15 & Nakijken code + werken aan verslag \\ \hline
    16 & Finaliseren verslag \\ \hline
    \end{tabular}
\end{center}

\end{document}
